% !TEX root = ../QChem_QFT_Book.tex
% --- Chapter 5: Algebraic Structure of Molecular Operators ---

\chapter{Algebraic Structure of Molecular Operators}
\label{chap:algebra}

Having defined the molecular creation and annihilation operators \(\Mdag{\alpha}\) and \(\Mop{\alpha}\) in \cref{chap:mol_ops}, we now examine their algebraic structure. The (anti)commutation relations between these operators govern the behavior of many-molecule systems, enable the description of chemical reactions, and encode the effective statistics and compositeness of molecules as quasi-particles.

A molecule composed of an even number of fermions behaves as a boson, while one with an odd number behaves as a fermion. This determines whether commutators \([A,B]\) or anticommutators \(\{A,B\}\) are used in describing the algebraic structure.

\section{(Anti)Commutation Relations}
\label{sec:algebra_commutation}

The fundamental operator relations for molecular operators begin with their action on the chemical reference state \(\RefState\):
\begin{align}
	\Mdag{\alpha} \RefState &= \ket{\Psi_\alpha}, \label{eq:mdag_action_recall} \\
	\Mop{\alpha} \ket{\Psi_\beta} &= \delta_{\alpha\beta} \RefState, \label{eq:m_op_action_recall} \\
	\Mop{\alpha} \RefState &= 0. \label{eq:m_op_on_ref}
\end{align}

These imply that, when acting on \(\RefState\), the canonical (anti)commutator yields:
\begin{itemize}
	\item For bosonic molecules:
	\[
	[\Mop{\alpha}, \Mdag{\beta}] \RefState = \delta_{\alpha\beta} \RefState.
	\]
	\item For fermionic molecules:
	\[
	\{\Mop{\alpha}, \Mdag{\beta}\} \RefState = \delta_{\alpha\beta} \RefState.
	\]
\end{itemize}

However, this result is only valid on \(\RefState\). It does not imply the operator identity:
\[
[\Mop{\alpha}, \Mdag{\beta}] = \delta_{\alpha\beta} \mathbb{I}
\quad \text{or} \quad
\{\Mop{\alpha}, \Mdag{\beta}\} = \delta_{\alpha\beta} \mathbb{I},
\]
as would hold for elementary bosonic or fermionic fields. The situation is more intricate for composite systems.

We also expect the following for operators of the same type:
\begin{align}
	\text{Bosons:} &\quad [\Mop{\alpha}, \Mop{\beta}] = 0, \quad [\Mdag{\alpha}, \Mdag{\beta}] = 0; \label{eq:boson_like_ops} \\
	\text{Fermions:} &\quad \{\Mop{\alpha}, \Mop{\beta}\} = 0, \quad \{\Mdag{\alpha}, \Mdag{\beta}\} = 0. \label{eq:fermion_like_ops}
\end{align}
These reflect symmetrization or antisymmetrization at the molecular level.

\section{Compositeness and Non-Canonical Behavior}
\label{sec:algebra_interpretation}

Although the canonical algebra appears valid on \(\RefState\), it fails more generally due to the composite structure of molecules. Consider:
\begin{equation}
	[\Mop{\alpha}, \Mdag{\beta}] \ket{\Psi_\gamma} = 
	\Mop{\alpha}(\Mdag{\beta} \ket{\Psi_\gamma}) 
	- \Mdag{\beta}(\Mop{\alpha} \ket{\Psi_\gamma}).
	\label{eq:comm_generic_state}
\end{equation}
The second term simplifies to \(-\delta_{\alpha\gamma} \ket{\Psi_\beta}\), but the first involves the action of \(\Mdag{\beta}\) on a molecular state, producing a multi-molecular state whose structure is complex and not reducible to simple \(\delta\)-functions.

This nontrivial behavior arises from:
\begin{itemize}
	\item \textbf{Pauli exclusion}: Fermionic constituents impose antisymmetry constraints even across distinct molecules.
	\item \textbf{Finite size}: Molecules are not point-like, and their wavefunctions can overlap.
	\item \textbf{Internal structure}: Composite systems carry internal correlations.
	\item \textbf{Intermolecular interactions}: Residual Coulomb and exchange forces affect overlapping states.
\end{itemize}

Hence, the true operator algebra is non-canonical. The deviations from \(\delta_{\alpha\beta}\) encode information about effective interactions and internal statistics.

\section{Approximate Canonical Behavior in Physical Limits}
\label{sec:algebra_approximations}

In many practical settings, the molecules are dilute and weakly interacting:
\begin{itemize}
	\item \emph{Low density}: \(\rho^{-1/3} \gg r_\text{mol}\), suppressing overlap.
	\item \emph{Low energy}: Internal structure is unresolved during scattering or field interaction.
\end{itemize}

Under these conditions, it becomes an excellent approximation to treat the molecules as effective elementary particles. We write:
\begin{align}
	\text{Bosonic molecules:} \quad
	&[\Mop{\alpha}, \Mdag{\beta}] \approx \delta_{\alpha\beta}, \\
	&[\Mop{\alpha}, \Mop{\beta}] \approx 0, \quad [\Mdag{\alpha}, \Mdag{\beta}] \approx 0; \\
	\text{Fermionic molecules:} \quad
	&\{\Mop{\alpha}, \Mdag{\beta}\} \approx \delta_{\alpha\beta}, \\
	&\{\Mop{\alpha}, \Mop{\beta}\} \approx 0, \quad \{\Mdag{\alpha}, \Mdag{\beta}\} \approx 0.
\end{align}

This approximation is widely used in:
\begin{itemize}
	\item Quantum gas theory (BECs, Fermi gases)
	\item Coarse-grained models of molecular reactivity
	\item Effective Hamiltonians in condensed-phase environments
\end{itemize}

\paragraph{Canonical vs. Non-Canonical Behavior.}
Although the operator algebra defined by \(\Mdag{\alpha}\) and \(\Mop{\alpha}\) superficially resembles that of elementary bosons or fermions, the underlying structure is more intricate due to the compositeness of molecular states. Unlike fundamental particles, molecules are constructed from bound clusters of fermions, and their creation operators are superpositions of many-body constituent configurations (cf. Eq.~\eqref{eq:mol_wavefunction}).

As a result, their operator algebra is inherently non-canonical in the full Hilbert space. For example, multi-molecule states created by repeated action of \(\Mdag{\alpha}\) do not generally remain orthogonal or normalized due to Pauli exclusion and finite spatial overlap of constituents.

The canonical (anti)commutation relations:
\[
[\Mop{\alpha}, \Mdag{\beta}] = \delta_{\alpha\beta}, \quad \text{or} \quad \{\Mop{\alpha}, \Mdag{\beta}\} = \delta_{\alpha\beta},
\]
hold only approximately, and only within a restricted subspace where:
\begin{itemize}
  \item The system is dilute (intermolecular separation \(\gg\) molecular size),
  \item The constituent wavefunctions have negligible overlap,
  \item Only low-multiplicity sectors (few excitations) are relevant.
\end{itemize}

This dilute limit justifies replacing the exact operator algebra with an effective canonical algebra in the EFT treatment (cf. Chapter~\ref{chap:eft}). Outside this regime, more careful composite field techniques or diagrammatic methods must be employed.

A related discussion in the context of excitons, Cooper pairs, and composite bosons appears in \cite{Combescot2008}, which also explores algebraic deviations from canonical forms.

\section{Symmetries and Conserved Quantities}
\label{sec:algebra_symmetries}

Each molecular operator \(\Mdag{\alpha}\) carries quantum numbers associated with conserved symmetries:
\begin{itemize}
	\item Momentum: \(\mathbf{P}\)
	\item Angular momentum: \(J, M\)
	\item Parity, nuclear spin, and other labels
\end{itemize}

Under a rotation \(R\), for instance, the operator \(\Mdag{J,M}\) must transform according to the irreducible representation \(D^J_{MM'}(R)\). The transformation properties are inherited from the internal structure of the operator as a composite of fundamental fields.

The full operator algebra must respect symmetry constraints imposed by the total Hamiltonian. In particular:
\begin{itemize}
	\item The total molecular number operator \(\hat{N}_{\text{mol}} = \sum_\alpha \Mdag{\alpha} \Mop{\alpha}\) is conserved in the absence of reactions.
	\item Hamiltonians constructed from \(\Mdag{\alpha}, \Mop{\alpha}\) must commute with the generators of continuous symmetries (e.g., translation, rotation) to ensure conservation laws are satisfied.
\end{itemize}

\section{Summary}

This chapter has examined the algebraic structure of molecular operators. While their action on the reference state mimics canonical relations, the internal compositeness and fermionic substructure of molecules lead to significant deviations from canonical algebra in general. These deviations encode nontrivial physical effects: exchange, overlap, interaction, and internal structure.

However, in dilute and low-energy regimes, molecules may be effectively treated as elementary particles with canonical statistics, enabling the use of simplified operator algebras in Effective Field Theory and many-body formulations.

In the next chapter, we leverage these insights to construct a Lagrangian-based Effective Field Theory where the molecular fields \(\Malpha(x)\) serve as the fundamental degrees of freedom.

