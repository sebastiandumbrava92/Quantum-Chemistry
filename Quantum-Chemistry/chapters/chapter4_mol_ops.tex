% !TEX root = ../QChem_QFT_Book.tex
% --- Chapter 4: Molecular Creation and Annihilation Operators ---

\chapter{Molecular Creation and Annihilation Operators}
\label{chap:mol_ops}

Building on the formal structures developed in \cref{chap:foundations,chap:qc_standard}, we now introduce the core constructs of this framework: the \emph{molecular creation operator} \(\Mdag{\alpha}\) and the \emph{molecular annihilation operator} \(\Mop{\alpha}\). These operators formalize the treatment of molecules as composite quasi-particles in the language of quantum field theory, offering a unified way to describe molecular states, interactions, and transformations within a second-quantized formalism.

\section{The Chemical Reference State \texorpdfstring{\(\RefState\)}{|Ref⟩}}
\label{sec:mol_ops_ref_state}

To define a molecular creation operator, we must specify the state from which a molecule is created. In QFT, the vacuum \(\vac\) is the lowest-energy state of the theory, containing no real particles. However, most chemical processes conserve baryon and lepton number and do not involve vacuum fluctuations or antiparticle production.

\begin{definition}[Chemical Reference State]
	\label{def:ref_state}
	The \emph{Chemical Reference State} \(\RefState\) is a multi-particle state containing the correct number of constituent particles (e.g., \(N_e\) electrons and relevant nuclei) to form a molecule, but in a non-interacting configuration corresponding to infinite spatial separation. It serves as the baseline state for constructing bound molecular states.
\end{definition}

This state is not the QFT vacuum \(\vac\) but rather a low-energy eigenstate of a free Hamiltonian describing separated particles. It conserves quantum numbers and provides a physically realistic reference for chemistry.

\section{Formal Construction of \texorpdfstring{\(\Mdag{\alpha}\)}{M†\_α}}
\label{sec:mol_ops_construction}

Let \(\ket{\Psi_\alpha}\) denote the exact eigenstate of a molecule in quantum state \(\alpha\), with
\begin{equation}
	\Ham \ket{\Psi_\alpha} = E_\alpha \ket{\Psi_\alpha}.
	\label{eq:mol_eigenstate}
\end{equation}
The label \(\alpha\) encompasses all relevant degrees of freedom: electronic configuration, vibrational and rotational quantum numbers, center-of-mass motion, and nuclear spin states.

\begin{definition}[Molecular Creation Operator]
	\label{def:molecular_creation}
	The molecular creation operator \(\Mdag{\alpha}\) is defined via its action on the reference state:
	\begin{equation}
		\Mdag{\alpha} \RefState = \ket{\Psi_\alpha},
		\label{eq:mol_creation_def}
	\end{equation}
	assuming that \(\RefState\) contains the requisite constituent particles and that the state is normalized.
\end{definition}

This is an operational definition: it specifies what \(\Mdag{\alpha}\) does, not how it is constructed microscopically. The operator \(\Mdag{\alpha}\) encodes all correlation and structure required to bind the constituents into the molecular state \(\ket{\Psi_\alpha}\).

\paragraph{Clarifying Note: Operational vs. Constructive Definitions}
The definition of the molecular creation operator \(\Mdag{\alpha}\) as satisfying
\begin{equation}
    \Mdag{\alpha} \RefState = \ket{\Psi_\alpha}
\end{equation}
should be understood in an \textbf{operational sense}: it identifies \(\Mdag{\alpha}\) as a black-box operator whose action produces the fully interacting molecular state \(\ket{\Psi_\alpha}\) from a non-interacting reference configuration. It does \textbf{not} imply an explicit constructive formula for \(\Mdag{\alpha}\) in terms of the constituent fields.

This abstraction is analogous to the role of interpolating fields in QCD, where composite operators like \(\bar{\psi}\gamma_5 \psi\) are used to represent bound states such as mesons without requiring their full Fock-space decomposition.

In this sense, \(\Mdag{\alpha}\) can be thought of as \textbf{encoding the full quantum chemical structure} of the molecule in its matrix elements rather than in its explicit form.

A simple toy model construction will be presented in Appendix~\ref{app:toy_model_mdag} to illustrate the underlying principle in a concrete setting.

\section{Properties of Molecular Operators}
\label{sec:mol_ops_properties}

We now explore structural properties of the operators \(\Mdag{\alpha}\) and \(\Mop{\alpha} = (\Mdag{\alpha})^\dagger\).

\paragraph{Adjoint and Annihilation:} The annihilation operator satisfies
\begin{equation}
	\Mop{\alpha} \ket{\Psi_\beta} = \delta_{\alpha\beta} \RefState,
	\qquad
	\Mop{\alpha} \RefState = 0.
	\label{eq:mol_annihilation}
\end{equation}
It removes a molecule in state \(\alpha\) and returns the system to the reference configuration. If the state does not contain \(\alpha\), the result is zero.

\paragraph{Linearity:} The operators are linear in the usual sense:
\[
\Mdag{\alpha}(c_1 \ket{\phi} + c_2 \ket{\psi}) = 
 c_1 \Mdag{\alpha} \ket{\phi} + c_2 \Mdag{\alpha} \ket{\psi}.
\]

\paragraph{Normalization:} Assuming orthonormality \(\braket{\Psi_\alpha}{\Psi_\beta} = \delta_{\alpha\beta}\), it follows that:
\[
\mel{\RefState}{\Mop{\alpha}}{\Psi_\beta} = \delta_{\alpha\beta}.
\]

\paragraph{Bosonic or Fermionic Statistics:} A molecule behaves as a composite boson or fermion depending on the parity of its total number of constituent fermions:
\begin{itemize}
	\item Even fermion count \(\Rightarrow\) Bosonic molecule.
	\item Odd fermion count \(\Rightarrow\) Fermionic molecule.
\end{itemize}
These statistics determine the expected algebra of the operators (commutation or anticommutation), which we will analyze in \cref{chap:algebra}.

\section{Example: The Hydrogen Molecule (H\textsubscript{2})}
\label{sec:mol_ops_examples}

Let us illustrate the above constructs with the hydrogen molecule.

\paragraph{Constituents:} 2 protons, 2 electrons \(\Rightarrow\) total fermion count = 4 \(\Rightarrow\) composite boson.

\paragraph{Reference State:}
\[
\RefState = \ket{p^+_1, p^+_2, e^-_1, e^-_2; \text{separated}},
\]
a state with constituents at infinite separation.

\paragraph{Target State:}
\[
\ket{\Psi_g} = \text{Ground state of H}_2,
\]
including electronic, vibrational, rotational, and nuclear spin degrees of freedom.

\paragraph{Creation Operator:}
\begin{equation}
	\Mdag{H_2, g} \RefState = \ket{\Psi_g}.
\end{equation}

\paragraph{Annihilation Operator:}
\begin{equation}
	\Mop{H_2, g} \ket{\Psi_g} = \RefState,
	\qquad
	\Mop{H_2, g} \RefState = 0.
\end{equation}

\paragraph{Excited States:} Each distinct excited state \(\alpha = e\) requires a separate operator \(\Mdag{\alpha}\). For instance, a vibrationally excited state or electronically excited state is created by \(\Mdag{H_2, e}\), with a different coefficient function \(C(\dots; \alpha)\) in \cref{eq:mol_wavefunction}.

\section{Outlook}

The molecular creation and annihilation operators provide an operator-theoretic description of molecular states as emergent excitations in a field-theoretic setting. Their formal definition parallels the interpolating operators of QCD and opens the door to analyzing molecular dynamics, statistics, and interactions algebraically.

In the next chapter, we will investigate the algebraic structure of these operators: their commutation or anticommutation relations, closure properties, and implications for constructing an effective quantum theory of molecular systems.

