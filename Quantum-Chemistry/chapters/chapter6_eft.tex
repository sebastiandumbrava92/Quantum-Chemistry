% !TEX root = ../QChem_QFT_Book.tex
% --- Chapter 6: Effective Field Theory (EFT) for Molecular Systems ---

\chapter{Effective Field Theory (EFT) for Molecular Systems}
\label{chap:eft}

Chapters~\ref{chap:mol_ops} and~\ref{chap:algebra} established the foundation for treating molecules as emergent composite particles via creation and annihilation operators \(\Mdag{\alpha}\), \(\Mop{\alpha}\). While the operator formalism captures their structure and statistics, it is less suited for constructing dynamical or statistical models directly. In this chapter, we formulate an \textbf{Effective Field Theory (EFT)} in which the molecular degrees of freedom are described by quantum fields, and their interactions are encoded in a systematically improvable Lagrangian.

\section{From Operators to Fields: Introducing \texorpdfstring{\(M_\alpha(x)\)}{Malpha(x)}}
\label{sec:eft_fields}

We promote the molecular operators \(\Mdag{\alpha}, \Mop{\alpha}\) to field operators \(M_\alpha(x)\), where \(x = (ct, \vb{x})\). These fields create and annihilate a molecule in internal state \(\alpha\) localized near spacetime point \(x\):
\begin{itemize}
	\item \(M^\dagger_\alpha(x)\): Creates a molecule in state \(\alpha\) at point \(x\).
	\item \(M_\alpha(x)\): Annihilates a molecule in state \(\alpha\) at point \(x\).
\end{itemize}

The field \(M_\alpha(x)\) inherits the overall statistics of the molecule:
\begin{itemize}
	\item Bosonic molecule \(\Rightarrow\) scalar or vector field.
	\item Fermionic molecule \(\Rightarrow\) spinor field.
\end{itemize}

For non-relativistic systems, we may instead work with Schrödinger fields \(\Psi_\alpha(t, \vb{r})\). The fields satisfy approximate equal-time (anti)commutation relations:
\begin{align}
	[M_\alpha(t, \vb{x}), M^\dagger_\beta(t, \vb{y})] &\approx \delta_{\alpha\beta} \delta^{(3)}(\vb{x} - \vb{y}), \\
	[M_\alpha(t, \vb{x}), M_\beta(t, \vb{y})] &\approx 0, \quad \text{(bosons)} \\
	\{M_\alpha(t, \vb{x}), M^\dagger_\beta(t, \vb{y})\} &\approx \delta_{\alpha\beta} \delta^{(3)}(\vb{x} - \vb{y}), \quad \text{(fermions)} \\
	\{M_\alpha(t, \vb{x}), M_\beta(t, \vb{y})\} &\approx 0.
\end{align}

These relations assume low density and weak overlap (as justified in Chapter~\ref{chap:algebra}) and are valid within the EFT's domain.

\section{Constructing the Effective Lagrangian \texorpdfstring{\(\mathcal{L}_{\mathrm{eff}}\)}{Leff}}
\label{sec:eft_lagrangian}

The dynamics of the molecular fields are governed by an effective Lagrangian density:
\[
\Lag_{\mathrm{eff}} = \sum_\alpha \Lag_{0,\alpha} + \Lag_{\mathrm{int}},
\]
where \(\Lag_{0,\alpha}\) describes free propagation and \(\Lag_{\mathrm{int}}\) encodes interactions.

\subsection{Free Field Terms}

For a bosonic molecule of mass \(m_\alpha\), a relativistic free Lagrangian is:
\begin{equation}
	\Lag_{0,\alpha} = \partial_\mu M^\dagger_\alpha \partial^\mu M_\alpha - m_\alpha^2 M^\dagger_\alpha M_\alpha. \label{eq:lag_free_scalar}
\end{equation}

In non-relativistic form:
\begin{equation}
	\Lag_{0,\alpha} = \ii M^\dagger_\alpha \partial_t M_\alpha - \frac{1}{2m_\alpha} \nabla M^\dagger_\alpha \cdot \nabla M_\alpha - V_{\text{int}}^\alpha M^\dagger_\alpha M_\alpha.
\end{equation}

Fermionic molecules are described using Pauli or Dirac Lagrangians as appropriate.

\subsection{Interaction Terms}

Typical interaction terms include:

\begin{itemize}
	\item \textbf{Elastic Scattering (Contact Interaction)}:
	\begin{equation}
		\Lag_{\text{int}}^{(4)} = -\frac{\lambda_{\alpha\beta\gamma\delta}}{N_{\text{sym}}} M^\dagger_\alpha M^\dagger_\beta M_\gamma M_\delta.
	\end{equation}
	
	\item \textbf{Chemical Reactions}:
	\begin{equation}
		\Lag_{\text{react}} = -\kappa M^\dagger_C M^\dagger_D M_A M_B - \kappa^* M^\dagger_A M^\dagger_B M_C M_D.
	\end{equation}
	
	\item \textbf{Internal Transitions}:
	\begin{equation}
		\Lag_{\text{trans}} = -g_{\alpha\beta} M^\dagger_\beta M_\alpha + \text{h.c.}
	\end{equation}
	
	\item \textbf{Electromagnetic Coupling (Minimal Coupling)}:
	\[
	\partial_\mu \rightarrow D_\mu = \partial_\mu + \ii q A_\mu.
	\]
	For dipole transitions:
	\begin{equation}
		\Lag_{\text{EM}} = -\vb{d}_{\alpha\beta} \cdot \vb{E}(x) M^\dagger_\beta M_\alpha.
	\end{equation}
\end{itemize}

The precise terms retained depend on the physical processes modeled.

\section{Matching to Quantum Chemistry and Experiment}
\label{sec:matching_to_qc}

The effective Lagrangian \( \Leff \) contains low-energy constants (LECs), including masses \( m_\alpha \), couplings \( \lambda_{\alpha\beta\gamma\delta} \), transitions \( \kappa \), and dipole elements \( d_{\alpha\beta} \). These parameters must be matched to either quantum chemistry calculations or experimental measurements.

\paragraph{Sources of Matching Data}
\begin{itemize}
  \item \textbf{Quantum Chemistry}:
  \begin{itemize}
    \item Bound state energies \( E_\alpha \Rightarrow m_\alpha \),
    \item Dipole matrix elements \( \mel{\Psi_\beta}{\hat{\mu}}{\Psi_\alpha} \Rightarrow d_{\alpha\beta} \),
    \item Scattering amplitudes \( \Rightarrow \lambda_{\alpha\beta\gamma\delta} \),
    \item Reaction transition matrix elements \( \Rightarrow \kappa \).
  \end{itemize}
  \item \textbf{Experiment}:
  \begin{itemize}
    \item Spectroscopic data: \( E_\alpha, d_{\alpha\beta} \),
    \item Reaction rates, cross-sections, phase shifts.
  \end{itemize}
\end{itemize}

\paragraph{Uncertainty Propagation}
Uncertainties in matching inputs propagate to EFT predictions:
\begin{itemize}
  \item Truncation errors in the EFT expansion,
  \item Statistical and systematic uncertainty in LECs,
  \item Combined error bars on observables.
\end{itemize}

\paragraph{Workflow Summary}
\[
\text{QC/Experiment} \longrightarrow \text{LECs} \longrightarrow \Leff \longrightarrow \text{EFT Predictions}.
\]
A schematic diagram is provided in Figure~\ref{fig:eft_matching_workflow}.
\begin{figure}[h]
	\centering
	\begin{tikzcd}[column sep=large, row sep=huge]
		\textbf{Quantum Chemistry} \quad \text{or} \quad \textbf{Experiment} \arrow[d, "\text{Matching}"'] \\
		\textbf{Low-Energy Constants (LECs):} \quad m_\alpha, \lambda_{\alpha\beta\gamma\delta}, \kappa, d_{\alpha\beta} \arrow[d, "\text{Insert into}"] \\
		\textbf{Effective Lagrangian} \quad \mathcal{L}_{\mathrm{eff}} \arrow[d, "\text{Used in}"] \\
		\textbf{EFT Predictions:} \quad \text{Scattering, Rates, Spectra, Thermodynamics}
	\end{tikzcd}
	\caption{Schematic workflow for matching EFT to quantum chemistry or experiment. The low-energy constants (LECs) extracted from microscopic data are inserted into the effective Lagrangian \( \mathcal{L}_{\mathrm{eff}} \), which enables predictive modeling of observables at the molecular scale.}
	\label{fig:eft_matching_workflow}
\end{figure}



\section{Power Counting and Validity}
\label{sec:eft_principles}

The EFT is valid below a cutoff scale \(\Lambda\), above which molecular compositeness cannot be neglected. The Lagrangian is organized in an expansion:
\[
\Lag_{\mathrm{eff}} = \Lag_{\text{LO}} + \Lag_{\text{NLO}} + \Lag_{\text{NNLO}} + \cdots
\]
Each term is suppressed by powers of \(E/\Lambda\), where \(E\) is the characteristic energy. Power counting allows:
\begin{enumerate}
	\item Identification of dominant contributions (LO).
	\item Systematic inclusion of higher-order corrections.
	\item Estimation of theoretical uncertainty.
\end{enumerate}

\section{Feynman Rules for Molecular EFT}
\label{sec:eft_feynman}

The effective Lagrangian generates Feynman rules for perturbative calculations:
\begin{itemize}
	\item \textbf{Propagator:}
	\[
	\frac{\ii}{p^2 - m_\alpha^2 + \ii\epsilon}.
	\]
	\item \textbf{Vertices:}
	\begin{itemize}
		\item 4-point: \( -\ii \lambda_{\alpha\beta\gamma\delta} / N_{\text{sym}} \),
		\item Reaction: \( -\ii \kappa \),
		\item EM coupling: \( -\ii q (p + p')^\mu \).
	\end{itemize}
	\item \textbf{External Lines:} Incoming/outgoing molecules.
	\item \textbf{Conservation:} Momentum conservation at each vertex.
\end{itemize}

\section{Optional: Renormalization Group Concepts}
\label{sec:eft_rg}

The renormalization group (RG) describes how EFT parameters evolve with scale \(\mu\). LECs such as \(\lambda, \kappa\) acquire scale dependence:
\[
\mu \frac{d \lambda}{d\mu} = \beta_\lambda(\lambda, \dots).
\]

This running allows:
\begin{itemize}
	\item Resummation of logarithmic corrections,
	\item Understanding universal behavior near criticality,
	\item Motivating the form of higher-order operators.
\end{itemize}

\section*{Summary}

This chapter introduced the Effective Field Theory formulation for molecular systems, promoting molecular operators to fields and constructing a Lagrangian consistent with symmetry, compositeness, and scale separation. This formalism enables systematic modeling of interactions, reactions, and transitions in complex molecular ensembles. We are now equipped to apply this framework to statistical mechanics and dynamical processes using tools such as path integrals and ensemble theory in the next chapter.
